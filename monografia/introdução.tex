\documentclass[12pt]{article}

%% Escrevendo em português
\usepackage[brazil]{babel}
\usepackage[utf8]{inputenc}
%\usepackage[latin1]{inputenc}
\usepackage[usenames,dvipsnames,svgnames,table]{xcolor}
\usepackage[a4paper,margin={1in}]{geometry}
\usepackage{graphicx}
\usepackage{indentfirst}
\usepackage{url}

%% Pulando linhas
\renewcommand{\baselinestretch}{2}

\newcommand{\vsp}{\vspace{0.2in}}
\newcommand{\etal}{{\it et al.}} 
\newcommand{\aluno}{{Carlos Eduardo Leão Elmadjian}}
\newcommand{\link}{\textit{link}} 
\newcommand{\links}{\textit{links}} 


\begin{document}
	
	\clearpage
	\thispagestyle{empty}
	
	\begin{minipage}[t]{6in}
		\begin{center}
			UNIVERSIDADE DE SÃO PAULO \\
			INSTITUTO DE MATEMÁTICA E ESTATÍSTICA\\
			DEPARTAMENTO DE CIÊNCIA DA COMPUTAÇÃO\\
			\vspace{5em}
			{\Large \textbf{Detecção de padrões de leitura por rastreadores de baixo desempenho}}\\
			\vspace{2em}
			{\large \textbf{Relatório parcial para a disiciplina MAC0499}}\\
			\vspace{2em} 
		\end{center}
		
		\vspace{8em}
		{\large \textbf{Aluno:} \aluno{}\\ }
		\vsp
		{\large \textbf{Orientador:} Prof. Dr. Carlos Hitoshi Morimoto\\}
		
		
	\end{minipage}
	
	\vspace{150pt}
	
	
	\newpage
	
	\vsp
	
	%================================================================	
	\section{Resumo}
	
	Neste trabalho, exploramos 
	
	%================================================================
	\section{Introdução}
	
		%------------------------------------------------------------
		\subsection{Motivação}
		
		
		%------------------------------------------------------------
		\subsection{Justificativa}
	
		%------------------------------------------------------------
		\subsection{Objetivos}
		
		%------------------------------------------------------------
		\subsection{Desafios}
	
	%================================================================	
	\section{Características da leitura}
	
		%------------------------------------------------------------
		\subsection{O movimento do olhar na leitura}
		
		%------------------------------------------------------------
		\subsection{Depreendimentos do comportamento de leitura}
		
		%------------------------------------------------------------
		\subsection{Diferenças entre leitura e \textit{skimming}}
		
		%------------------------------------------------------------
		\subsection{Como reconhecer a leitura por máquina}
		
	%================================================================
	\section{Algoritmos}
	
		%------------------------------------------------------------
		\subsection{Algoritmo de Campbell e Maglio (2001)}
		
		%------------------------------------------------------------
		\subsection{Algoritmo de Buscher et al. (2008)}
		
		%------------------------------------------------------------
		\subsection{Algoritmo de Kollmorgen e Holmqvist (2007)}
		
	%================================================================
	\section{Estudo da baixa amostragem}
		
		%------------------------------------------------------------
		\subsection{A frequência de Shannon-Nyquist}
		
		%------------------------------------------------------------
		\subsection{Padrões do olhar e o número de amostras}
		
		%------------------------------------------------------------
		\subsection{Resolução mínima para detecção da pupila}
		
		%------------------------------------------------------------
		\subsection{Estudo das soluções}
	
	%================================================================	
	\section{Novo algoritmo}
		
		%------------------------------------------------------------
		\subsection{Concepção}
		
		%------------------------------------------------------------
		\subsection{Implementação}
		
		%------------------------------------------------------------
		\subsection{Testes e metodologia}
		
		%------------------------------------------------------------
		\subsection{Resultados}
		
		%------------------------------------------------------------
		\subsection{Discussão}
		
	%================================================================
	\section{Prova de conceito}
	
		%------------------------------------------------------------
		\subsection{Idealização e tentativas iniciais}
		
		%------------------------------------------------------------
		\subsection{Descrição do software}
		
		%------------------------------------------------------------
		\subsection{Possibilidades e futuras aplicações}
		
	%================================================================
	\section{Conclusões}
	
		%------------------------------------------------------------
		\subsection{Eliminando o \textit{aliasing} na detecção da leitura}
		
		%------------------------------------------------------------
		\subsection{Comunicação passiva como forma de interação}
		
		%------------------------------------------------------------
		\subsection{Dificuldades a serem superadas}
		
		
\bibliographystyle{plain}
\bibliography{references} 

\end{document}